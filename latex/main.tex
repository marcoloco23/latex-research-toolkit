\documentclass[12pt]{article}

% Basic setup
\usepackage[utf8]{inputenc}
\usepackage[T1]{fontenc}
\usepackage{lmodern}
\usepackage{geometry}
\geometry{margin=1in}
\usepackage{setspace}
\onehalfspacing

% Math and symbols
\usepackage{amsmath,amssymb,amsthm}
\usepackage{mathtools}

% Graphics and tables
\usepackage{graphicx}
\usepackage{booktabs}
\usepackage{array}

% Bibliography and references
\usepackage[backend=biber,style=alphabetic,sorting=nyt]{biblatex}
\addbibresource{references.bib}

% Hyperlinks (load last)
\usepackage{hyperref}
\hypersetup{
    colorlinks=true,
    linkcolor=blue,
    citecolor=red,
    urlcolor=blue
}

% Custom macros
%======================================================================
%  Global notation macros — safe definitions (use %======================================================================
%  Global notation macros — safe definitions (use %======================================================================
%  Global notation macros — safe definitions (use \input{macros.tex})
%======================================================================)
%======================================================================)
%======================================================================

% Document information
\title{Research Paper Title}
\author{Author Name}
\date{\today}

\begin{document}

\maketitle

\begin{abstract}
This is the abstract of your research paper. Briefly describe the problem, methods, results, and conclusions. Keep it concise but informative, typically 150-250 words.
\end{abstract}

\tableofcontents
\newpage

% MODULAR SECTIONS - Edit individual files in sections/ directory
% This approach allows AI to focus on one section at a time

% Uncomment to use modular sections:
% % Introduction Section - Focused editing for AI assistance
% Use: % Introduction Section - Focused editing for AI assistance
% Use: % Introduction Section - Focused editing for AI assistance
% Use: \input{sections/introduction} in main.tex

\section{Introduction}

This section demonstrates how to split your main document into focused sections for better AI collaboration.

\subsection{Research Problem}
State your research problem clearly. This focused approach allows AI to:
\begin{itemize}
    \item Concentrate on one section at a time
    \item Provide more targeted suggestions
    \item Maintain consistency within each section
    \item Reduce context overload
\end{itemize}

\subsection{Motivation}
Explain why this research is important and what gap it fills in the literature.

\subsection{Contributions}
List your main contributions:
\begin{enumerate}
    \item First contribution
    \item Second contribution 
    \item Third contribution
\end{enumerate}

\subsection{Paper Organization}
Briefly outline the structure of the remaining paper.  in main.tex

\section{Introduction}

This section demonstrates how to split your main document into focused sections for better AI collaboration.

\subsection{Research Problem}
State your research problem clearly. This focused approach allows AI to:
\begin{itemize}
    \item Concentrate on one section at a time
    \item Provide more targeted suggestions
    \item Maintain consistency within each section
    \item Reduce context overload
\end{itemize}

\subsection{Motivation}
Explain why this research is important and what gap it fills in the literature.

\subsection{Contributions}
List your main contributions:
\begin{enumerate}
    \item First contribution
    \item Second contribution 
    \item Third contribution
\end{enumerate}

\subsection{Paper Organization}
Briefly outline the structure of the remaining paper.  in main.tex

\section{Introduction}

This section demonstrates how to split your main document into focused sections for better AI collaboration.

\subsection{Research Problem}
State your research problem clearly. This focused approach allows AI to:
\begin{itemize}
    \item Concentrate on one section at a time
    \item Provide more targeted suggestions
    \item Maintain consistency within each section
    \item Reduce context overload
\end{itemize}

\subsection{Motivation}
Explain why this research is important and what gap it fills in the literature.

\subsection{Contributions}
List your main contributions:
\begin{enumerate}
    \item First contribution
    \item Second contribution 
    \item Third contribution
\end{enumerate}

\subsection{Paper Organization}
Briefly outline the structure of the remaining paper. 
% % Methodology Section - Focused editing for AI assistance
% Use: % Methodology Section - Focused editing for AI assistance
% Use: % Methodology Section - Focused editing for AI assistance
% Use: \input{sections/methodology} in main.tex

\section{Methodology}

\subsection{Experimental Setup}
Describe your experimental design and setup.

\subsection{Data Collection}
Explain how data was collected:
\begin{itemize}
    \item Data sources
    \item Collection methods
    \item Quality assurance measures
\end{itemize}

\subsection{Analysis Methods}
Detail your analytical approach:

\begin{equation}
    f(x) = \argmax_{y \in \mathcal{Y}} P(y|x, \theta)
\end{equation}

Where $\theta$ represents the model parameters.

\subsection{Evaluation Metrics}
List the metrics used to evaluate performance:
\begin{align}
    \text{Precision} &= \frac{TP}{TP + FP} \\
    \text{Recall} &= \frac{TP}{TP + FN}
\end{align}  in main.tex

\section{Methodology}

\subsection{Experimental Setup}
Describe your experimental design and setup.

\subsection{Data Collection}
Explain how data was collected:
\begin{itemize}
    \item Data sources
    \item Collection methods
    \item Quality assurance measures
\end{itemize}

\subsection{Analysis Methods}
Detail your analytical approach:

\begin{equation}
    f(x) = \argmax_{y \in \mathcal{Y}} P(y|x, \theta)
\end{equation}

Where $\theta$ represents the model parameters.

\subsection{Evaluation Metrics}
List the metrics used to evaluate performance:
\begin{align}
    \text{Precision} &= \frac{TP}{TP + FP} \\
    \text{Recall} &= \frac{TP}{TP + FN}
\end{align}  in main.tex

\section{Methodology}

\subsection{Experimental Setup}
Describe your experimental design and setup.

\subsection{Data Collection}
Explain how data was collected:
\begin{itemize}
    \item Data sources
    \item Collection methods
    \item Quality assurance measures
\end{itemize}

\subsection{Analysis Methods}
Detail your analytical approach:

\begin{equation}
    f(x) = \argmax_{y \in \mathcal{Y}} P(y|x, \theta)
\end{equation}

Where $\theta$ represents the model parameters.

\subsection{Evaluation Metrics}
List the metrics used to evaluate performance:
\begin{align}
    \text{Precision} &= \frac{TP}{TP + FP} \\
    \text{Recall} &= \frac{TP}{TP + FN}
\end{align} 
% \input{sections/results}
% \input{sections/discussion}
% \input{sections/conclusion}

% For now, using simplified inline sections for demonstration:

\section{Introduction}

This is the introduction section. Here you should:
\begin{itemize}
    \item Introduce the research problem
    \item Provide background and motivation
    \item State your contributions
    \item Outline the paper structure
\end{itemize}

You can use citations like this \cite{exampleref} and mathematical expressions like $f(x) = x^2 + 1$.

\section{Literature Review}

Discuss related work and how your research fits into the existing literature. Compare and contrast different approaches.

\section{Methodology}

Describe your research methods, experimental setup, or theoretical framework. Use mathematical notation when appropriate:

\begin{equation}
    \mathbb{E}[X] = \sum_{i=1}^{n} x_i P(X = x_i)
\end{equation}

You can also use the custom macros defined in `macros.tex`:
- Real numbers: $\R$
- Natural numbers: $\N$
- Argmax operator: $\argmax_{x} f(x)$

\section{Results}

Present your findings here. Include tables and figures as needed:

\begin{table}[h]
\centering
\begin{tabular}{@{}lcc@{}}
\toprule
Method & Accuracy & Runtime \\
\midrule
Algorithm A & 95.2\% & 10.3s \\
Algorithm B & 92.8\% & 5.7s \\
\bottomrule
\end{tabular}
\caption{Comparison of different algorithms}
\label{tab:results}
\end{table}

\section{Discussion}

Interpret your results, discuss limitations, and suggest future work. You can reference tables like Table~\ref{tab:results}.

\section{Conclusion}

Summarize your main findings and contributions. Restate the significance of your work.

% Examples of TODO and NOTE commands from macros.tex
\todo{Remember to add more experimental results}
\note{Consider discussing computational complexity}

\printbibliography

\end{document}